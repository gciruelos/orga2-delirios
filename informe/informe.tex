\documentclass[hidelinks,a4paper,12pt, nofootinbib]{article}
\usepackage[a4paper, left=3cm, top=3cm, right=2cm, bottom=2cm]{geometry}
\usepackage[spanish, es-tabla]{babel} %es-tabla es para que ponga Tabla en vez de Cuadro en el caption
\usepackage[utf8]{inputenc}
\usepackage[T1]{fontenc}
\usepackage{xspace}
\usepackage{xargs}
\usepackage{fancyhdr}
\usepackage{lastpage}
\usepackage{caratula}
\usepackage[bottom]{footmisc}
\usepackage{amsmath}
\usepackage{amssymb}
\usepackage{algorithm}
\usepackage[noend]{algpseudocode}
\usepackage{array}
\usepackage[dvipsnames]{xcolor,colortbl}
\usepackage{amsthm}
\usepackage{listings}
% Si no esta esto se rompe todo: http://tex.stackexchange.com/questions/202614
\lccode`~=0
\usepackage{soul}
\usepackage{mathpazo}

\usepackage{pgf}

\usepackage{graphicx}
\usepackage{sidecap}
\usepackage{amsmath}
\usepackage{wrapfig}
\usepackage{caption}

\lstdefinestyle{customc}{
  belowcaptionskip=1\baselineskip,
  breaklines=true,
  xleftmargin=\parindent,
  language=C,
  morekeywords={
    int8_t, int16_t, int32_t, int64_t,
    uint8_t, uint16_t, uint32_t, uint64_t,
    dir_entry
  },
  showstringspaces=false,
  basicstyle=\footnotesize\ttfamily,
  keywordstyle=\bfseries\color{blue!40!black},
  commentstyle=\itshape\color{purple!40!black},
  identifierstyle=\color{black},
  stringstyle=\color{orange},
}

\lstdefinestyle{customasm}{
  belowcaptionskip=1\baselineskip,
  frame=L,
  xleftmargin=\parindent,
  language=[x86masm]Assembler,
  basicstyle=\footnotesize\ttfamily,
  commentstyle=\itshape\color{purple!40!black},
}



%Formato de los links
\usepackage{hyperref}
\hypersetup{
  colorlinks   = true, %Colours links instead of ugly boxes
  urlcolor     = blue, %Colour for external hyperlinks
  linkcolor    = blue, %Colour of internal links
  citecolor   = red %Colour of citations
}

\usepackage{comment}

%Bibliografia
\usepackage[
  backend=bibtex,
  style=alphabetic
]{biblatex}
\addbibresource{bibliografia.bib}


\captionsetup[table]{labelsep=space}


\setlength{\parindent}{4em}
\setlength{\parskip}{0.5em}


%%fancyhdr
\pagestyle{fancy}
\thispagestyle{fancy}
\addtolength{\headheight}{1pt}
\lhead{Organización del Computador II}
\rhead{Gonzalo Ciruelos Rodríguez}
\cfoot{\thepage\ / \pageref{LastPage}}
\renewcommand{\footrulewidth}{0.4pt}
\renewcommand{\labelitemi}{$\bullet$}

\newenvironment{puntos}{%
    \itemize
    \color{Red}
}{%
    \enditemize
}

%%caratula
\materia{Organización del Computador II}
\titulo{Trabajo Práctico Final}

\fecha{\today}

\usepackage{etoolbox}
\AtBeginEnvironment{tikzpicture}{\shorthandoff{>}\shorthandoff{<}}{}{}

\begin{document}
\maketitle

\tableofcontents
\newpage

\section{Introducción}
DeliriOS es un exokernel bare-metal cuyo objetivo es proveer una base de desarrollo para programas donde ocurren muchas operaciones de sincronización entre threads.

El objetivo original de este trabajo era implementar, sobre el trabajo de Sebastián Nale y Julián Pinelli, el módulo de escritura para ext2.
Luego, el trabajo derivó en otros asuntos que fueron resualtos, como el diseño de la API de IO de DeliriOS (hecho en conjunto con Sebastián y Julián) y la refactorización del driver para discos rígidos (IDE) y la detección de dispositivos PCI (requerida para la refactorización del driver).

Todos los objetivos fueron cumplidos, y el objetivo de este trabajo es explicarlos, documentarlos y exhibir resultados generales sobre l
\newpage

\section{DeliriOS}

\subsection{Trabajo previo}
\begin{enumerate}
  \item Trabajo de noit y Goldsmith.
\end{enumerate}

\subsection{Motivación de este trabajo}

\begin{enumerate}
  \item Por qué un filesystem
  \item Por qué ext2.
\end{enumerate}


\newpage

\section{ext2}

\subsection{El sistema de archivos ext2}
\begin{enumerate}
  \item Historia e idea general de ext2.
  \item Descripcion detallada de ext2.
\end{enumerate}

\subsection{Otras implementaciones}
\begin{enumerate}
  \item Breve descripción de como otros sistemas operativos resolvieron el problema de los filesystems en general y ext2 en particular.
\end{enumerate}

\subsection{Nuestra implementación}
\begin{enumerate}
  \item Descripción de la API de IO de delirios.
  \item Overview general de la implementación.
  \item Descripción detallada, funcion por funcion de la implementacion.
  \item Como testeamos, unit testing y testing e2e.
\end{enumerate}


\newpage

\chapter{Dispositivos}

\section{PCI}

El bus PCI (Peripheral Component Interconnect) fue definido para establecer un bus local de alta performance y bajo costo que durara varias generaciones de productos.
Combinando un camino transparente de mejoras desde 132 MB/s (32-bit a 33MHz) hasta 528 MB/s (64-bit a 66MHz) y ambientes de señalización tanto de 5V como de 3.3V, el bus PCI satiface las necesidades tanto de computadoras de escritorio de bajo costo como de servidores LAN de alto costo.
El bus PCI es independiente del procesador, lo cual permite una transición eficiente a futuros procesadores, además de poder usar diferentes arquitecturas de procesadores.

Las desventajas del bus PCI es que es la limitada cantidad de cargas eléctricas que puede conducir. Un único bus PCI puede conducir un máximo de 10 cargas. (Recordemos que cuando contamos la cantidad de cargas de un bus, un conector cuenta como una carga y el dispositivo PCI cuenta como una carga, o a veces dos).

Cualquier sistema operativo que se precie de serlo debe poder detectar que dispositivos están conectados a la computadora. Por esta razón, decidimos implementar un brevisimo identificador de dispositivos PCI.

Este es un paso vital para trabajar en el futuro con cualquier tipo de dispositivo, por ejemplo USB o Ethernet.

\subsection{Cómo detectar dispositivos PCI}

La especificación del bus PCI nos provee de un sistema de inicialización y configuración que puede realizarse totalmente por software via un espacio de direcciones de configuración separado para cada dispositivo en el bus PCI.

Todos los dispositivos PCI, excepto el host bridge, están obligados a proveer 256 bytes de registros de configuración para ese propósito

Los ciclos de configuración de los ciclos de lectura y escritura son usados para acceder al espacio de configuración de cada dispositivo objetivo. El objetivo es seleccionado enviando una señal ISDEL al host bridge. La señal ISDEL actua como la clásica señal "chip select". Durante la fase de direcciones del ciclo de configuración, el procesador puede direccionar alguno de los 64 registros de 32-bits dentro del espacio de configuración del dispositivo, simplemente escribiendo el número de registro requerido en las lineas de direcciones 2 a 7 (AD[7..2]) y las lineas de byte enable.

Los dispositivos PCI son inherentemente little endian, lo que significa que todos los campos de más de un byte tienen los valores menos significativos en las direcciones más bajas. Esto requiere que cuando se programa un procesador big-endian, como un Power PC, se hagan las operaciones apropiadas de byte-swapping al leer y escribir de la memoria del dispositivo. Sin embargo, como DeliriOS es un sistema operativo para arquitectura AMD64, que es little-endian, nada de esto nos interesará.


\subsection{Detección de dispositivos PCI en DeliriOS}

La función principal del módulo PCI (llamada en kernel.c, durante la inicialización del sistema) es la siguiente.

\begin{lstlisting}[style=customcmucho]
void pci_check_buses(void) {
     uint8_t header_type = pci_read_config(get_device(0, 0, 0), HEADER_TYPE);
     if ((header_type & 0x80) == 0){
         // Un solo PCI host controller
         check_bus(0);
     } else {
         // Multiples PCI host controllers. Tengo que chequear las funciones
         // (0, 0, f) . Si esta up, quiere decir que hay un host controller ahi
         for (uint8_t function = 0; function < 8; function++) {
             if (pci_read_config(get_device(0, 0, function), VENDOR_ID) != 0xFFFF)
                 break;
             check_bus(function);
         }
     }
}
\end{lstlisting}

Primero nos fijamos cuantos PCI host controllers hay. En caso de que sea uno solo, chequeo el bus 0. En caso de que sean muchos, tengo que chequear la función de cada bus, porque si está UP quiere decir que ahi hay un host controller.


Pasemos a ver el resto de las funciones. 

\begin{lstlisting}[style=customcmucho]
static inline uint16_t get_device(uint16_t bus, uint16_t slot, uint16_t func){
    return ((bus << 8) | (slot << 3) | func);
}
\end{lstlisting}

Esta función nos permite generar el código de un dispositivo, dado su bus, función y slot.


\begin{lstlisting}[style=customcmucho]
uint32_t pci_read_config(uint16_t device, uint16_t field_and_size){
    // Armo el pedido para informacion del dispositivo
    uint8_t field = field_and_size >> 8;
    uint8_t size  = field_and_size & 0x00ff;

    uint32_t address = ((uint32_t) ((device << 8) | (field & 0xfc))) 
                       | 0x80000000;
 
    // Pido esa entrada de la informacion del dispositivo
    outl(0xCF8, address);

   	if (size == 4) {
		uint32_t t = inl(0xCFC);
		return t;
	} else if (size == 2) {
		uint16_t t = inw(0xCFC + (field & 2));
		return t;
	} else if (size == 1) {
		uint8_t t = inb(0xCFC + (field & 3));
		return t;
	} 
    return 0xFFFF;
}
\end{lstlisting}

Esta función nos permite leer un field del espacio de configuración de un dispositivo. Field and size es un parámetro que contiene tanto el field como el tamaño de ese field.

Para leer el field, utilizamos dos puertos de 32-bits. El primero es 0xCF8, que se llama CONFIG\_ADDRESS, y el segundo es 0xCFC, que se llama CONFIG\_DATA. CONFIG\_ADDRESS especifica la dirección del espacio de configuración que vamos a querer acceder, mientras que escrituras sucesivas a CONFIG\_DATA van a hacer generar el acceso al espacio de configuración del dispositivo PCI.

\begin{lstlisting}[style=customcmucho]
static char * lookup_device(uint8_t class, uint8_t subclass){
    for(uint64_t i = 0; i < DEVICE_CLASSES; i++){
        if(device_info[i].class == class && device_info[i].subclass == subclass)
            return device_info[i].info;
    }
    return device_info[DEVICE_CLASSES - 1].info;
}
\end{lstlisting}

Esta función se ocupa de, dada una clase y una sublcase, devolver un string que contiene una descripción del dispositivo (en inglés). La tabla se encuentra al final de esta sección.


Las funciones que siguen son todas las funciones que usa pci\_check\_buses para probar todas las combinaciones posibles de slot, funcion y bus.

\begin{lstlisting}[style=customcmucho]
static void check_slot(uint8_t bus, uint8_t slot) {
     uint16_t device = get_device(bus, slot, 0);
 
     if (pci_read_config(device, VENDOR_ID) == 0xFFFF)
         return; // No existe el dispositivo
     check_function(device);
     uint8_t header_type = pci_read_config(device, HEADER_TYPE);

     if ((header_type & 0x80) != 0){
         // Es un dispositivo multifuncion, 
         // entonces chequeamos todas las funciones
         for(uint8_t function = 1; function < 8; function++) {
             uint16_t device = get_device(bus, slot, function);
             
             if(pci_read_config(device, VENDOR_ID) != 0xFFFF)
                 check_function(device);
         }
     } 
}
\end{lstlisting}

\begin{lstlisting}[style=customcmucho]
static void check_bus(uint8_t bus) { 
     for (uint8_t slot = 0; slot < 32; slot++)
         check_slot(bus, slot);
}
\end{lstlisting}

La función que sigue es la más complicada de las tres, dado que es la última, por lo tanto la que hace todo el trabajo. Aquí es donde confirmamos la existencia del dispositivo, y donde imprimimos a salida estándar una descripción de él.

En caso de que el dispositivo sea un disco IDE, lo inicializamos con la funcion ide\_init, que analizaremos en la siguiente sección.

\begin{lstlisting}[style=customcmucho]
void check_function(uint16_t device) {
     uint8_t class = pci_read_config(device, CLASS);
     uint8_t subclass = pci_read_config(device, SUBCLASS);
     uint16_t device_id = pci_read_config(device, DEVICE_ID);
     uint16_t vendor_id = pci_read_config(device, VENDOR_ID);
     uint8_t prog_if = pci_read_config(device, PROG_IF);
     uint8_t irq = pci_read_config(device, PROG_IF);

     pci_devices[next_pci_device++] = (struct pci_device) 
                                      {device_id, vendor_id, class, 
                                       subclass, prog_if, irq};

     // Imprimo toda la inforamcion sobre el dispositivo
     printf("  (PCI DEVICE) (%d,%d,%d)\t", 
             device >> 8, (device >> 3) & 31, device & 3);

     char* info = lookup_device(class, subclass);
     if (info)
         printf(info);
     else
         printf("0x%x 0x%x", class, subclass);

     printf(" - [0x%x 0x%x]\n", vendor_id, device_id);
     // Aca termino de imprimir la informacion sobre el dispositivo
     // Y procedo a registrarlo como corresponda
     
     if (class == 0x06 && subclass == 0x04)
         check_bus(pci_read_config(device, SECONDARY_BUS));
     if (class == 0x01 && subclass == 0x01){
         ide_init(pci_read_config(device, BAR0),
                  pci_read_config(device, BAR1),
                  pci_read_config(device, BAR2),
                  pci_read_config(device, BAR3),
                  pci_read_config(device, BAR4));
     }
}
\end{lstlisting}


Por último, veamos la lista de información de dispositivos que tenemos, que nos va a servir para saber que es un dispositivo dada su clase y sublase.

\begin{lstlisting}[style=customcmucho]
static struct class_info device_info[] = {
    {0x01, 0x01, "IDE Controller"},
    {0x01, 0x05, "ATA Controller"},
    {0x01, 0x06, "Serial ATA"},
    {0x02, 0x00, "Ethernet Controller"},
    {0x03, 0x00, "VGA-Compatible Controller"},
    {0x06, 0x00, "Host Bridge"},
    {0x06, 0x01, "ISA Bridge"},
    {0x06, 0x80, "Other Host Device"},
    {0x0C, 0x03, "USB Controller"},
    {0xFF, 0x00, NULL},
};

#define DEVICE_CLASSES (sizeof(device_info)/sizeof(struct class_info))
\end{lstlisting}



\newpage

\section{HDD}
\begin{enumerate}
  \item Descripcion general de la interfaz IDE.
  \item Que habia antes y que cambié.
\end{enumerate}

\subsection{IDE}
IDE es una sigla que se refiere a la especificación eléctrica de los cables que conectan unidades ATA con otro dispositivo. Las unidades usan la interfaz ATA (Advanced Technology Attachment). Un cable IDE generalmente termina siendo conectado a un controlador IDE, que a su vez está conectado a un bus PCI.

Cada dispositivo IDE aparece como un dispositivo en el bus PCI. Si el código de clase es 0x01 (Mass Storage Controller) y el código de subclase es 0x1 (IDE), entonces el dispositivo en cuestión es una unidad IDE. Una unidad IDE sólo utiliza cinco registros BAR de los seis disponibles.

\begin{enumerate}
 \item[BAR0] Dirección base del canal primario (espacio de I/O). Si es 0x0 o 0x1, entonces el puerto es 0x1F0.
 \item[BAR1] Dirección base del puerto de control del canal primario (espacio de I/O). Si es 0x0 o 0x1, entonces el puerto es 0x3F6.
 \item[BAR2] Dirección base del canal secundario (espacio de I/O). Si es 0x0 o 0x1, entonces el puerto es 0x170.
 \item[BAR3] Dirección base del puerto de control del canal secundario (espacio de I/O). Si es 0x0 o 0x1, entonces el puerto es 0x376.
 \item[BAR4] Bus Master IDE. Se refiere ala base del rango de I/O consistiendo de 16 puertos. Los primeros 8 puertos controlan el DMA del canal primario y los segundos ocho del secundario.
\end{enumerate}

Un controlador de IDE que funcione en paralelo utilizará las IRQ 14 y 15; un IDE serial solo utiliza un IRQ. Para leer esta IRQ, hay que mirar el espacio de configuración del dispositivo PCI, como fue explicado en la sección anterior.

\subsection{ATA PIO}
En la parte anterior vimos como detectar dispositivos de almacenamientoATA y como comunicarnos con ellos. Ahora vemos como escribir y leer de ellos. Esto ya estaba implementado en DeliriOS, pero de una forma incorrecta. Mi trabajo en esta parte se basó en hacer la implementación correcta, conforme al estándar.

Esta forma se basa en usar el comando IDENTIFY, que nos permitirá obtener más información del dispositivo.


PIO es una manera de leer y escribir a dispositibos ATA usando polling. Veamos de que se trata.

De acuerdo con la especificación de ATA, el modo PIO debe estar soportado por todas las unidades ATA como el mecanismo de transferencia de datos por default.

PIO, al ser un método de polling, usa una cantidad tremenda de recursos de CPU, pues cada byte de datos transferido entre el disco y el CPU debe ser enviado a atraves del puerto IO del procesador. 

\subsubsection{Más detalles de ATA PIO}

La especificación de ATA está hecha sobre otra más vieja llamada ST506. Con ST506, cada disco estaba conectado a una placa controladora por dos cables, un cable de datos y un cable de comandos.

La placa controladora estaba enchufada al bus de la placa madre. El procesador se comunicaba con la placa controladora a través de los puertos de IO del CPU, que estaban directamente conectados al bus del motherboard.

Lo que la especificación original de IDE hacía era despegar las placas controladores del motherboard, y pegar un controlador en cada disco rígido, permanentemente. Cuando el procesador accede al puerto IO del disco, hay un chip que se sirve de atajo entre los pins del bus IO del procesador y el cable IDE, por lo tanto el procesador podía acceder directamente a la placa controladora de la unidad IDE. El mecanismo de transferencia de datos entre el CPU y la controladora permaneció el mismo, y hoy es llamado modo PIO. 

\subsubsection{IDENTIFY}

Hasta aquí describimos lo que ya estaba en DeliriOS, que de hecho se remonta a un sistema operativo diseñado por Juan Pablo Darago (juampiOS). En este trabajo mejoramos ese código, modernizandolo y haciendolo obediente al estándar.

El comando IDENTIFY es la mejor forma de obtener información sobre los discos ATA conectados a la placa madre.

Para usar el comando IDENTIFY, primero hay que seleccionar la unidad objetivo enviando 0xA0 para la unidad maestro o 0xB0  para la unidad esclavo al puerto IO DRIVE\_SELECT. Luego, deben ponerse en 0 los puertos SECTOR\_COUNT, LBALO, LBAMID, LBAHI (es decir, los puertos 0x1F2 a 0x1F5). Luego, debe enviarse el comando IDENTIFY (0xEC) al puerto IO comando (0x1F7). Finalmente leer el puerto status (0x1F7) nuevamente. Si el valor leido es 0, la undiad no existe. Para cualquier otro valor, hay que hacer polling del puerto status hasta que el bit 7 (BSY) se ponga en 0.

En ese momento, si ERR está en 0, la información esta lista para ser leida del puerto de datos (0x1F0). Hay que leer 256 valores de 16 bits y guardarlos.

Alguna información interesante devuelta por IDENTIFY:

\begin{itemize}
 \item uint16\_t 0: es útil si el dispositivo no es un disco rígido (lectora de discos removibles, por ejemplo).
 \item uint16\_t 83: el bit 10 está seteado si el dispositivo soporta LBA48 mode (el modo de direccionamiento de sectores que utiliza DeliriOS)..
 \item uint16\_t 88: los bits en el byte bajo nos dicen si se soportan modos UDMA y el byte alto nos dice cuál modo UDMA está activo. 
 \item uint16\_t 60 y 61: tomados como un uint32\_t contiene el numero total de  sectores LBA de 28 bits direccionables en el disco (Si es distinto de cero, el disco soporta LBA28).
 \item uint16\_t 60 y 61: tomados como un uint32\_t contiene el numero total de  sectores LBA de 28 bits direccionables en el disco (Si es distinto de cero, el disco soporta LBA28).
 \item uint16\_t 100 a 103: tomados como una uint64\_t contiene el número total de sectores LBA de 48 bits direccionables en el disco.
\end{itemize}







\newpage

\chapter{Resultados}

\section{Evaluación de nuestra implementación}

En este momento vamos a evaluar debilidades y fortalezas de nuestra implementación de ext2. 

La debilidad obvia, y que no nos importa, es que la implementación no es completa: carece de manejo de permisos de archivos, cosa que no nos importa en DeliriOS.

Además, es mucho más simple que la implementación de ext2 de otros sistemas operativos como Linux, y usa algoritmos mucho menos sofisticados. Sin embargo, no esperaremos que su performance sea mucho peor. Esto lo veremos más adelante.

Una métrica no siempre utilizada al comparar varias implementaciones de una misma especificación es su simplicidad. Aquí sí lo haremos, dado que DeliriOS es un sistema operativo que se concentra en ser minimal, aunque funcional.

Comparemos la cantidad de líneas de código de cada implementación.

\begin{figure}[H]
  \centering
\begin{tikzpicture}
\begin{axis}[
	%x tick label style={/pgf/number format/1000 sep=},
  symbolic x coords={DeliriOS, Linux, FreeBSD, GNU Hurd},
	ylabel=LOC,
  xtick=data,
  ymin=0,
	ybar,
]
\addplot 
	coordinates {
    (DeliriOS,876)
    (Linux,5899)
    (FreeBSD,5910)
    (GNU Hurd,4104)};
\end{axis}
\end{tikzpicture}
\caption{Líneas de código de la implementación de ext2 de cada sistema operativo. El programa \texttt{cloc} fue utilizado para medirlas, dado que cuenta las lineas de código puro, sin contar líneas vacías y de comentarios.}
\end{figure}

\begin{figure}[H]
  \centering
\begin{tikzpicture}
\begin{axis}[
	%x tick label style={/pgf/number format/1000 sep=},
  symbolic x coords={DeliriOS, Linux, FreeBSD, GNU Hurd},
	ylabel=KB,
  xtick=data, 
  ymin=0,
	ybar,
]

%(DeliriOS,52477)
%(Linux,231427)
%(FreeBSD,232832)
%(GNUHurd,164757)};
\addplot 
	coordinates {
    (DeliriOS,51.24)
    (Linux,226.00)
    (FreeBSD,227.37)
    (GNU Hurd,160.89)};
\end{axis}
\end{tikzpicture}
\caption{Tamaño del código fuente de la implementación de ext2 de cada sistema operativo.}
\end{figure}


Como puede verse, la implementación de DeliriOS es realmente simple y corta. Esto es muy bueno, dado que por un lado es más fácil de entender para alguien que se acople al proyecto, y por otro lado permite que el binario de DeliriOS siga siendo diminuto (entra entero en la cache L1 de código de un core de un procesador moderno).



\section{Performance de nuestra implementación}

Para medir la performance de nuestra implementación decidimos compararla con la de Linux.
Tenemos que aclarar que medir este tipo de piezas de software y compararlas es muy difícil por varias razones.
Por ejemplo, como son operaciones dentro de todo muy rápidas, por lo que si las mediciones están mal hechas, el \emph{preemption} del sistema operativo las puede arruinar.

Como DeliriOS sólo soporta ATA PIO por ahora, necesitamos algún ambiente de prueba en el que esto no sea un limitante. Por eso, testeamos sobre una máquina virtual que emulara este tipo de disco.

Las mediciones las tomamos de la siguiente manera: linkeamos los mismos tests que los tests de correctitud por fuera de delirios y los usamos para testear distintas operaciones sobre una imagen de disco.
Tambi\'en realizamos las mismas operaciones con Linux.
Esto ya es un problema, porque el código de DeliriOS va a correr en \emph{userspace}, mientras que el código de Linux va a correr en \emph{kernelspace}.
Veremos cómo corregir esto más adelante, pero primero veamos cómo testeamos. Realizamos 500 repeticiones de cada test, y cada test consistió en lo siguiente.

\begin{itemize}
    \item[create] 36 archivos creados.
    \item[mkdir] 10 directorios creados.
    \item[read] 104000 caract\'eres leídos de 2 archivos.
    \item[remove] 8 archivos borrados.
    \item[write] 104000 caract\'eres escritos en 2 archivos.
\end{itemize}

Todos los tests fueron corridos con el correspondiente flush a disco (fsync en Linux) de tal manera que sean efectivos. Además, con DeliriOS medimos el tiempo de acceso a disco, y se lo restamos a ambas mediciones. Es decir, los tiempos que veremos no son los tiempos totales, si no los tiempos totales sin los tiempos que el programa se pasa leyendo o escribiendo el disco.

Veamos los resultados.


\begin{figure}[H]
  \centering
  \begin{minipage}[b]{0.49\textwidth}
    \includegraphics[width=\textwidth]{tiempos/create.pdf}
    \caption{}
  \end{minipage}
  \begin{minipage}[b]{0.49\textwidth}
    \includegraphics[width=\textwidth]{tiempos/mkdir.pdf}
    \caption{}
  \end{minipage}
  \hfill
\end{figure}
\begin{figure}[H]
  \centering
  \begin{minipage}[b]{0.49\textwidth}
    \includegraphics[width=\textwidth]{tiempos/read.pdf}
    \caption{}
  \end{minipage}
  \begin{minipage}[b]{0.49\textwidth}
    \includegraphics[width=\textwidth]{tiempos/remove.pdf}
    \caption{}
  \end{minipage}
\end{figure}
\begin{figure}[H]
  \centering
  \includegraphics[width=0.5\textwidth]{tiempos/write.pdf}
  \caption{}
\end{figure}

Como se ve, Linux es múcho más rápido en la mayoría de las funciones, excepto en remove. Creemos que la diferencia en remove se debe a que remove es una operación extremadamente barata, y Linux tiene muchas estructuras que debe mantener en cache (como índices), entonces actualizarlas le lleva la diferencia de tiempo que se ve.

Sin embargo, como dijimos antes, los tests están lejos de ser exactos, dado que DeliriOS corre en \emph{userspace} pero Linux corre en \emph{kernelspace}. Pero, como se puede ver en \cite{Rajgarhia:2010:PEU:1774088.1774130}, un trabajo muy extenso que analiza diferencias de performance entre filesystems que corren en el \emph{kernelspace} vs. en el \emph{userspace}, la diferencia llega a ser de 600\% de pérdida de performance.

Hacer la experimentación en DeliriOS nativo no es lo mejor para este tipo de experimentos, porque el clock con el que medimos los tiempos no tiene la granularidad suficiente.
Esto se debe a que usamos el RTC (Real Time Clock), que tiene una granularidad máxima de 122 us.
Por esta razón, solo corrimos el test más exigente, que es el de escritura. Este test, al igual que el anterior, es el tiempo de corrida total, menos el tiempo de acceso a disco.
Veamos los resultados.

\begin{figure}[H]
  \centering
  \includegraphics[width=0.5\textwidth]{tiempos/delirios_linux.pdf}
  \caption{Tiempos en DeliriOS vs. Linux en la prueba de escritura. Cada uno fue corrido en una máquina virtual de iguales condiciones, y a cada tiempo se le restó el tiempo de acceso a disco medido en cada caso. En el caso de Linux, medimos el tiempo de acceso a disco con delirios corriendo sobre Linux, como explicamos antes.}
\end{figure}

Las diferencias de performance ahora son mucho más razonables, dado que el código de DeliriOS es mucho más simple. Sin embargo, obviamente el código de Linux es mucho mas sofisticado, y funciona mejor seguramente en discos más llenos, por ejemplo, dado que sus algoritmos de reserva de bloques son mucho más complejos.

La varianza de Delirios es mucho mas alta, dado que la granularidad es muy baja, por lo que explicamos anteriormente.


\newpage

\section{Conclusión}

\subsection{Trabajo futuro}

\newpage

% si se descomenta esto, aparecen todas las cosas de la bibliografia, hasta
% las que nunca fueron citadas en el TP. es una eleccion de diseño.
\nocite{*}
\printbibliography

\end{document}

